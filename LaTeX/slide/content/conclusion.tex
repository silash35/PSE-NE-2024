\section{Considerações finais}

\begin{frame}{Considerações finais}
  \begin{itemize}
    \item O Arduino demonstrou capacidade para executar Redes Neurais Recorrentes fisicamente informadas, com ampla margem de recursos (apenas \textbf{68,4\%} da memória flash e \textbf{13,2\%} da memória RAM foram utilizados).
    \item Ao simular ruído nas medições, ou na ausência de medições, a PIRNN mantêm a qualidade das previsões relativamente boas, mostrando a robustez do modelo.
    \item Usando o onnxruntime, a PIRNN foi mais de três vezes mais rápida que os métodos numéricos tradicionais no mesmo computador.
  \end{itemize}
\end{frame}
